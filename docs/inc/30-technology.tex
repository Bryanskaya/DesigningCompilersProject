\section{Технологическая часть}
\subsection{Выбор средств программной реализации}
В качестве языка программирования была выбрана Java 17, ввиду нескольких причин.
\begin{itemize}
	\item Компилятор, написанный на Java, может быть запущен на различных платформах, поддерживающих виртуальную машину JVM.
	
	\item Предоставляется много библиотек и инструментов для разработки компиляторов, включающих в себя инструменты для анализа синтаксиса, генерации кода и оптимизации.
	
	\item В дополнение, на момент реализации уже был накоплен существенный опыт в
	использовании этого языка программирования. \\
\end{itemize}

\subsection{Сгенерированные классы анализаторов}
В результате работы ANTLR генерируются следующие файлы.
\begin{enumerate}
	\item \underline{Oberon.interp} и \underline{OberonLexer.interp} содержат данные (таблицы предсказания, множества следования, информация о правилах грамматики и т.д.) для интерпретатора ANTLR, используются для ускорения работы сгенерированного парсера для принятия решений о разборе входного потока.
	
	\item \underline{Oberon.tokens} и \underline{OberonLexer.tokens} перечислены символические имена токенов, каждому из которых сопоставлено числовое значение типа токена. ANTLR4 использует их для создания отображения между символическими именами токенов и их числовыми значениями.
	
	\item Интерфейсы \underline{OberonListener.java} и \underline{OberonVisitor.java}.
	
	\item \underline{OberonBaseListener.java} и \underline{OberonBaseVisitor.java} -- реализации соответствующих интерфейсов паттернов Listener и Visitor.
	
	\item \underline{OberonParser.java} -- синтаксический анализатор.
	
	\item \underline{OberonLexer.java} -- лексический анализатор. \\
\end{enumerate}

\subsection{Тестирование}
Для проверки корректной работы программы был написан класс OberonCompilerApplicationTests, в котором указаны пары файлов (исходный файл программы на языке Oberon и файл с ожидаемым результатом). При запуске тестирования последовательно обрабатывается каждый исходный файл и результат сравнивается с тем, что написан в соответствующем парном файле.\\

\subsection{Пример работы программы}
На листингах 1-2 ниже приведены примеры кода на языке Oberon для нахождения пятого числа Фибоначчи и соответствующие файлы (листинг 3-4) промежуточного представления. 

\lstinputlisting[caption = Программа для нахождения числа Фибоначчи \label{lst:fib-param}]{code/fib\_param.txt}

\lstinputlisting[caption = Файл промежуточного представления для программы нахождения чисел Фибоначчи \label{lst:fib-param-ll}]{code/fib\_param\_ll.txt}

\lstinputlisting[caption = Программа для нахождения числа Фибоначчи на массиве \label{lst:fib-array}]{code/fib\_array.txt}

\lstinputlisting[caption = Файл промежуточного представления для программы нахождения чисел Фибоначчи на массиве ]{code/fib\_array\_ll.txt}\label{lst:fib-array-ll}