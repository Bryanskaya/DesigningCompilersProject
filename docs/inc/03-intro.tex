\section*{ВВЕДЕНИЕ}
\addcontentsline{toc}{section}{ВВЕДЕНИЕ}

Согласно ежегодному прогнозу, который опубликовала Международная корпорация данных (IDC, International Data Corporation \cite{IDC}), занимающаяся мониторингом количества информации, объём созданных и воспроизведённых \, данных в 2020 году резко вырос \cite{IDC2020}.

Установлено, что этот показатель достиг 64.2 Збайт ($10^{21}$ байт) данных, что примерно в 2 раза больше, чем в 2018 году, когда отметка достигла 33 Збайт \cite{IDC2020,IDC2018}. Также предполагается, что в период с 2020 по 2025 годы объём будет продолжать активно расти \cite{IDC2025}.

В связи с этим задача поиска необходимой информации в больших массивах данных обостряется с каждым годом всё больше. Проблема усугубляется ещё и тем, что один и тот же объект может описываться по-разному, также стоит учитывать, что одна группа людей его может употреблять в ином контексте, нежели другая, всё это создаёт дополнительные сложности в обработке текстов на естественном языке. 

Целью данной выпускной квалификационной работы является разработка и реализация метода определения объекта из ограниченной выборки по \, нечёткому описанию на русском языке.

Для достижения цели необходимо решить следующие задачи:
\begin{itemize}	
	\item изучить основные алгоритмы компьютерной лингвистики и обосновать выбор тех, которые будут использованы для модернизации;
	
	\item сформировать выборку респондентов для формирования исходного перечня терминов;
	
	\item разработать метод определения объекта из ограниченной выборки по \, нечёткому описанию на русском языке;
	
	\item разработать программное обеспечение и протестировать его;
	
	\item оценить работоспособность разработанного метода и дать рекомендации о его применимости.
\end{itemize}


\pagebreak




















