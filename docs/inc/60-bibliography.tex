\section*{СПИСОК ИСПОЛЬЗОВАННЫХ ИСТОЧНИКОВ}
\addcontentsline{toc}{section}{СПИСОК ИСПОЛЬЗОВАННЫХ ИСТОЧНИКОВ}

\begingroup
\renewcommand{\section}[2]{}
\begin{thebibliography}{}
	\bibitem{bib:compilerIS} АХО А.В, ЛАМ М.С., СЕТИ Р., УЛЬМАН Дж.Д. Компиляторы: принципы, технологии и инструменты. – М.: Вильямс, 2008.
	
	\bibitem{bib:lex} Lesk M. E., Schmidt E. Lex: A lexical analyzer generator. – Murray Hill, NJ : Bell Laboratories, 1975. – С. 1-13.
	
	\bibitem{bib:flex} Sampath P. et al. How to test program generators? A case study using flex //Fifth IEEE International Conference on Software Engineering and Formal Methods (SEFM 2007). – IEEE, 2007. – С. 80-92.
	
	\bibitem{bib:antlr4} What is ANTLR? [Электронный ресурс]. -- Режим доступа: https://www.antlr.org/ (Дата обращения: 25.04.2023).
	
	\bibitem{bib:bison} Donnelly C. BISON the YACC-compatible parser generator //Technical report, Free Software Foundation. – 1988.
	
	\bibitem{bib:llvm} The LLVM Compiler Infrastructure Project [Электронный ресурс]. -- Режим доступа: https://llvm.org/ (Дата обращения: 27.04.2023).
	
	
	
	
	
	
	
	
	
	
	
	
	
	
	
	
	
	
	
\end{thebibliography}
\endgroup

\pagebreak