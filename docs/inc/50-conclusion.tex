\section*{ЗАКЛЮЧЕНИЕ}
\addcontentsline{toc}{section}{ЗАКЛЮЧЕНИЕ}
Таким образом, в рамках текущей выпускной квалификационной работы был разработан и реализован метод определения объекта из ограниченной выборки по нечёткому описанию на естественном языке. Объём проделанной работы соответствует требованиям технического задания. 

Разработанное приложение позволяет: 
\begin{itemize}
	\item определять объект из ограниченной выборки по нечёткому описанию на естественном языке, запрос может быть изложен как через текстовое поле, так и через голосовой ввод;
	
	\item обновлять из интерфейса онтологию, построенную как на статистических данных, так и на графовых структурах;
	
	\item наглядно демонстрировать строение синтаксических графов, как в составе сети, так и по отдельности.
\end{itemize}
%
В результате проделанной работы были выполнены все поставленные задачи.
\begin{itemize}
	\item Была проанализирована предметная область, проведён сравнительный \, анализ существующих методов решения, выявлены основные преимущества и недостатки.
	
	\item Также, рассмотрены особенности работы с текстами на естественном \, языке, обоснована необходимость в предварительной обработке.
	
	\item Описан принцип формирования онтологии и основные ограничения для определения ключевых слов. 
	
	\item В результате проведённого предварительного анализа, определены основные этапы поиска нечётких дубликатов, а также критерий принятия решения об использовании вспомогательного метода.
	
	\item Формализованы входные и выходные данные метода. 
	
	\item Пошагово описана структура реализуемого алгоритма.
	
	\item Разработано и протестировано программное обеспечение, демонстрирующее работу данного метода.
	
	\item Проведено исследование поведения алгоритма при различных входных данных и онтологиях.
\end{itemize}
%
В качестве направлений дальнейшей работы можно выделить следующие:
\begin{itemize}
	\item дальнейшее увеличение датасета;
	
	\item использование параллельных вычислений при нахождении косинусного сходства запроса и составляющих онтологии;
	
	\item определение критерия досрочного выхода из процедуры поиска в сети синтаксических графов в целях уменьшения времени обработки запроса.
\end{itemize}


\pagebreak